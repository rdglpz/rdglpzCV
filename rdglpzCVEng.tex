\documentclass[7pt]{article}
\usepackage[utf8]{inputenc}


% This is a helpful package that puts math inside length specifications
\usepackage{multirow}
\usepackage{wasysym}
\usepackage{marginnote}
\usepackage{url}
\usepackage{calc}
\usepackage[spanish, es-tabla]{babel}
\usepackage{graphicx}
\usepackage{xcoffins}
\usepackage{array}
\usepackage{multicol}
\usepackage{multirow}
\usepackage{titlesec}
\usepackage[scaled=.90]{helvet}
\usepackage{fancyhdr}
\usepackage[ddmmyyyy,hhmmss]{datetime}
\titleformat{\section}
  {\normalfont\sffamily\Large\bfseries\color{cyan}}
  {\thesection}{1em}{}
 




%\usepackage[latin1]{inputenc}


% Layout: Puts the section titles on left side of page
\reversemarginpar

%
%         PAPER SIZE, PAGE NUMBER, AND DOCUMENT LAYOUT NOTES:
%
% The next \usepackage line changes the layout for CV style section
% headings as marginal notes. It also sets up the paper size as either
% letter or A4. By default, letter was used. If A4 paper is desired,
% comment out the letterpaper lines and uncomment the a4paper lines.
%
% As you can see, the margin widths and section title widths can be
% easily adjusted.
%
% ALSO: Notice that the includefoot option can be commented OUT in order
% to put the PAGE NUMBER *IN* the bottom margin. This will make the
% effective text area larger.
%
% IF YOU WISH TO REMOVE THE ``of LASTPAGE'' next to each page number,
% see the note about the +LP and -LP lines below. Comment out the +LP
% and uncomment the -LP.
%
% IF YOU WISH TO REMOVE PAGE NUMBERS, be sure that the includefoot line
% is uncommented and ALSO uncomment the \pagestyle{empty} a few lines
% below.
%

%% Use these lines for letter-sized paper
\usepackage[paper=letterpaper,
            %includefoot, % Uncomment to put page number above margin
            marginparwidth=1.0in,     % Length of section titles
            marginparsep=.05in,       % Space between titles and text
            margin=1in,               % 1 inch margins
            includemp]{geometry}

%% Use these lines for A4-sized paper
%\usepackage[paper=a4paper,
%            %includefoot, % Uncomment to put page number above margin
%            marginparwidth=30.5mm,    % Length of section titles
%            marginparsep=1.5mm,       % Space between titles and text
%            margin=25mm,              % 25mm margins
%            includemp]{geometry}

%% More layout: Get rid of indenting throughout entire document
\setlength{\parindent}{0in}

%% This gives us fun enumeration environments. compactitem will be nice.
\usepackage{paralist}


%% Reference the last page in the page number
%
% NOTE: comment the +LP line and uncomment the -LP line to have page
%       numbers without the ``of ##'' last page reference)
%
% NOTE: uncomment the \pagestyle{empty} line to get rid of all page
%       numbers (make sure includefoot is commented out above)
%
\usepackage{fancyhdr,lastpage}
\pagestyle{fancy}
%\pagestyle{empty}      % Uncomment this to get rid of page numbers
\fancyhf{}\renewcommand{\headrulewidth}{0pt}
\fancyfootoffset{\marginparsep+\marginparwidth}
\newlength{\footpageshift}
\setlength{\footpageshift}
          {0.5\textwidth+0.5\marginparsep+0.5\marginparwidth-2in}
\lfoot{\hspace{\footpageshift}%
       \parbox{4in}{\, \hfill %
                    \arabic{page} / \protect\pageref*{LastPage} % +LP
%                    \arabic{page}                               % -LP
                    \hfill \,}}

% Finally, give us PDF bookmarks
\usepackage{color,hyperref}
\definecolor{darkblue}{rgb}{0.0,0.0,0.3}
\hypersetup{colorlinks,breaklinks,
            linkcolor=darkblue,urlcolor=darkblue,
            anchorcolor=darkblue,citecolor=darkblue}

%%%%%%%%%%%%%%%%%%%%%%%% End Document Setup %%%%%%%%%%%%%%%%%%%%%%%%%%%%


%%%%%%%%%%%%%%%%%%%%%%%%%%% Helper Commands %%%%%%%%%%%%%%%%%%%%%%%%%%%%

% The title (name) with a horizontal rule under it
%
% Usage: \makeheading{name}
%
% Place at top of document. It should be the first thing.
\newcommand{\makeheading}[1]%
        {\hspace*{-\marginparsep minus \marginparwidth}%
         \begin{minipage}[t]{\textwidth+\marginparwidth+\marginparsep}%
                {\large \bfseries #1}\\[-0.15\baselineskip]%
                 \rule{\columnwidth}{1pt}%
         \end{minipage}}

% The section headings
%
% Usage: \section{section name}
%
% Follow this section IMMEDIATELY with the first line of the section
% text. Do not put whitespace in between. That is, do this:
%
%       \section{My Information}
%       Here is my information.
%
% and NOT this:
%
%       \section{My Information}
%
%       Here is my information.
%
% Otherwise the top of the section header will not line up with the top
% of the section. Of course, using a single comment character (%) on
% empty lines allows for the function of the first example with the
% readability of the second example.
\renewcommand{\section}[2]%
        {\pagebreak[2]\vspace{1.3\baselineskip}%
         \phantomsection\addcontentsline{toc}{section}{#1}%
         \hspace{0in}%
         \marginpar{
         \raggedright \scshape #1}#2}

% An itemize-style list with lots of space between items
\newenvironment{outerlist}[1][\enskip\textbullet]%
        {\begin{itemize}[#1]}{\end{itemize}%
         \vspace{-.6\baselineskip}}

% An environment IDENTICAL to outerlist that has better pre-list spacing
% when used as the first thing in a \section 
\newenvironment{lonelist}[1][\enskip\textbullet]%
        {\vspace{-\baselineskip}\begin{list}{#1}{%
        \setlength{\partopsep}{0pt}%
        \setlength{\topsep}{0pt}}}
        {\end{list}\vspace{-.6\baselineskip}}

% An itemize-style list with little space between items
\newenvironment{innerlist}[1][\enskip\textbullet]%
        {\begin{compactitem}[#1]}{\end{compactitem}}

% To add some paragraph space between lines.
% This also tells LaTeX to preferably break a page on one of these gaps
% if there is a needed pagebreak nearby.
\newcommand{\blankline}{\quad\pagebreak[2]}

%%%%%%%%%%%%%%%%%%%%%%%% End Helper Commands %%%%%%%%%%%%%%%%%%%%%%%%%%%

%%%%%%%%%%%%%%%%%%%%%%%%% Begin CV Document %%%%%%%%%%%%%%%%%%%%%%%%%%%%

\newcolumntype{M}[1]{>{\centering\arraybackslash}m{#1}}
\newcolumntype{P}[1]{>{\hfill\arraybackslash}p{#1}}

\begin{document}
%{\sffamily 
%\marginnote{typeset text here...}
%{\fontfamily{cmss}\selectfont
\small{
\makeheading{\begin{tabular}{l P{70mm}}
Ph.D. Rodrigo L\'{o}pez Far\'{i}as         		&	\footnotesize{\textit{\today}  }				
\end{tabular}}  



\section{Personal Information} 
\begin{tabular}{l P{70mm} M{20mm}}
Birthday: 8-Jul-1984		&	e-mail: rodrigo.lopez@alumni.imtlucca.it	& \multirow{4}{*}[1cm]{\includegraphics[width=2.2cm]{foto.png}  } 	\\
Address: Aram\'{e}n \# 313		&   ORCID ID: 0000-0003-2772-0051			&	\\
Postal (ZIP) Code: 58070		& 	Cellular: +52 443 155 5416				&	\\
Morelia, Mexico         		&	Skype ID: rdglpz  						&
\end{tabular}


\blankline

\section{Interests \& Skills  } \textbf{Programming Languages} 

MATLAB, MATHEMATICA, R, Java, C/C++, PHP-HTML-MySQL, Python, LISP.

\blankline

\textbf{Research} 


Machine learning, data mining, dimensionality reduction, time series, nonlinear dynamical systems, global optimization, evolutionary computing.
 
\blankline

\textbf{Research groups and projects}

Mexican Center of Energy Innovation Project.% Project: Forecasting the Natural Resources Required for the Production of Renewable Electric Power.

Applied Computational Intelligence Network.


\blankline
 
\textbf{Languages} 

English: 550 ITP TOEFL points. 

Italian: B1 Common CEFRL Level. 


\section{Academic Degree}  \textbf{Ph.D in Computer Science and Engineering}. (W. European Doctorate mention).

IMT School of Advanced Studies Lucca. Lucca, Italy. \textit{(Feb-2012 Jan-2016)}. 

Thesis: Time Series Forecasting Based on Classification of Dynamic Patterns.

Advisors: Dr. Alberto Bemporad. Dr. Pantelis Sopasakis.

Field of study: Time series analysis.

\blankline

\textbf{MSc in Electrical Engineering (Computer Systems Group)}.

Univesidad Michoacana de San Nicolas de Hidalgo. Morelia, Mexico. \textit{(Mar-2008 Aug-2010)}.

Thesis: Bifurcation Diagrams for Discontinuous or Non-differentiable Equations.

Advisors:  Dr. Juan Jose Flores Romero, Dr. Claudio Fuerte E.

Field of study:  Evolutionary computing, nonlinear dynamical systems, stability analysis and optimization.

\blankline

\textbf{B.Eng. in Computer Systems}.

Instituto Tecnológico de Morelia. Morelia, México. \textit{(2002 2007)}.

Thesis: Implementation and performance analysis of \textbf{``Linux Terminal Server Project"} for educational purposes.

Field of Study: Applications of distributed operative systems.

\section{Academic Experience}  \textbf{Teaching}.


\blankline
\textbf{Instituto Tecnol\'{o}gico de Morelia}. Morelia, Mexico.

\begin{innerlist}
\item Structured programming and object oriented programming (Electronic and  Industrial Engineering), Research Methodology (Computational Systems Engineering). \textit{(Aug-2011 Jan-2012)}.
\item Database Fundamentals (Computational Systems Engineering), Structures and organization of data. (Technology Information Engineering) and Evaluation of software projects. \textit{(Jan-2011 Jul-2011)}.
\item Operative systems, selected topics in programming and research fundamentals. \textit{(Aug-2010 Dec-2010)}.
\end{innerlist}


\blankline

\blankline
\textbf{Universidad de Morelia}. Morelia, Mexico.

\begin{innerlist}
\item Web programming with PHP. \textit{(Aug-2009 Dec-2009)}.
\end{innerlist}

\blankline

\section{Professional Experience}
%\subsection{Participation in Research Projects}
\textbf{State Center for Information and Communications Technologies (CETIC)}. \textit{(Mar-2007 Jun-2007)}.
Morelia, Mexico.

Resident in physical infrastructure department.

Project: Performance analysis of \textbf{Linux Terminal Server Project} applied to to basic education.

\blankline

{\textbf{Instituto Tecnologico de Morelia}}. Morelia, Mexico. \textit{(Feb-2007)}

Social Service Project: Develop of a PHP Web catalog for Social Service.

\blankline

\textbf{IMPULSA}
\blankline

Participation in the young entrepreneurs program: IMPULSA.


\blankline

\section{Publications} \textbf{Journal Articles in JCR}

\blankline
\textbf{Accepted}

%\subsection{Journal Articles Sent}
\begin{innerlist}
\item \textit{Hector Rodriguez Rangel, Vicen\c{c} Puig, Rodrigo L\'{o}pez Far\'{i}as, Juan J. Flores }.  Short-Term Demand Forecast using Bank of Neural Network Models Trained using Genetic Algorithms for the Optimal Management of Drinking Water Networks.  \textit{Journal Hydroinformatics}. DOI: 10.2166/hydro.2016.199. ISSN: 1464-7141 (2016)
%\item \textit{Juan J. Flores, Rodrigo L\'{o}pez Far\'{i}as, Julio Barrera, Carlos Coello }.  Performance of Gravitational Interaction Optimization Metaheuristics on High Dimensional Problems.  \textit{Journal of Metaheuristics}.(2016)
%\item \textit{Juan J. Flores, Jos\'{e} Cede\~no Gonzalez, Rodrigo L\'{o}pez Far\'{i}as, F\'{e}lix Calder\'{o}n }.  Evolving Nearest Neighbor Time Series Forecasters. \textit{Journal of Time Series Analysis}.(2016)
\end{innerlist}
\blankline



\textbf{Refereed Conference Papers}

\blankline
\textbf{Accepted}
\begin{innerlist}


\item \textit{Juan J. Flores, Felix Calderon Solorio, Jose Rafael Cede\~no Gonzalez, Jose Ortiz Bejar and Rodrigo Lopez Farias}.  Comparison of Time Series Forecasting Techniques with respect to Tolerance to Noise \textit{IEEE Autumn Meeting on Power, Electronics and Computing}, \textbf{Ixtapa M\'{e}xico, November 2016 }


\item \textit{Hector Rodriguez-Rangel, Vicen\c{c} Puig, Juan J. Flores and ,  Rodrigo L\'{o}pez} Far\'{i}as.  Flow meter Data Validation and Reconstruction using Neural Networks: Application to the Barcelona Water Network \textbf{3rd International Conference on Control and Fault-Tolerant Systems, Barcelona, Spain. 2016}.

\item \textit{Hector Rodriguez Rangel, Vicen\c{c} Puig,Juan J. Flores and ,  Rodrigo L\'{o}pez Far\'{i}as.  Flow meter Data Validation and Reconstruction using Neural Networks: Application to the Barcelona Water Network} \textbf{2016 European Control Conference, Aalborg, Denmark. June 2016}.

\item \textit{Rodrigo L\'{o}pez Far\'{i}as, Juan J. Flores and Vicen\c{c} Puig}.  Qualitative and Quantitative Multi-Model Forecasting with Nonlinear Noise Filter Applied to Water Demand \textit{IEEE Autumn Meeting on Power, Electronics and Computing}. DOI: 10.1109/ROPEC.2015.7395122.  \textbf{Ixtapa M\'{e}xico, November 2015}.

\item \textit{Juan J. Flores, Jose Ortiz Bejar, Jose Rafael Cedeno, Carlos Lara-Alvarez and Rodrigo L\'{o}pez Far\'{i}as} FNN a Fuzzy Version of the Nearest Neighbour Time Series Forecasting Technique \textit{IEEE Autumn Meeting on Power, Electronics and Computing} \textbf{Ixtapa M\'{e}xico, November 2015 }.

\item \textit{Rodrigo L\'{o}pea Far\'{i}as, Vicen\c{c} Puig}   A Multiple-Model Predictor Approach Based on an On-Line Mode Recognition with Application to Water Demand Forecasting \textit{International work-conference on Time Series 1 
} \textbf{Granada Spain, July 2015}.

\item \textit{Rodrigo. L\'{o}pez, Vicen\c{c} Puig, Hector Rodriguez}   An implementation of a multi-model predictor based on the qualitative and quantitative decomposition of the time-series \textit{International work-conference on Time Series 1 
} \textbf{Granada Spain, July 2015}.

\item \textit{Dr, Juan Flores, Rodrigo López, Julio Barrera.} Optimization with gravitational Interactions  \textit{ROPEC XIII: Autumn Meeting of Electric power systems, electronic and computation ( Reuni\'{o}n de Oto\~no de Potencia, Electr\'{o}inca y Computaci\'{o}n)} \textbf{ Morelia M\'{e}xico, November 2011}.

\item Juan Flores, Rodrigo Lopez, Julio Barrera. Gravitational Interactions Optimization. In \textit{Learning and Intelligent OptimizatioN}  (LION 5) \textbf{Rome, Italy - January 2011}.

\item Juan J. Flores, Rodrigo Lopez and Julio Barrera. Particle swarm optimization with gravitational interactions for multimodal and unimodal problems. In \textit{Proceedings of the 9th Mexican International Conference on Artificial Intelligence (MICAI 2010)}, pages 3361-370. Springer-Verlag. \textbf{Pachuca, México. November 2010.}

\end{innerlist}

\section{Conferences, Seminars \& Workshops } \textbf{Given} \begin{innerlist}
 %\item Learning and Intelligent OptimizatioN - 'Gravitational Interactions Optimization'. (Rome, Italy. January 2011)
\item IV National Seminar of computer learning and intelligence (SNAIC). Water demand prediction with Genetic Algorithms for the optimum optimization of a Drinking Water Distribution System. Instituto Nacional de Astrofísica, Óptica y Eléctrónica. Universidad Michoacana de San Nicolás de Hidalgo. (Morelia, México. September 2016).
 \item 10mo Congreso Estatal de Ciencia, Tecnolog\'{i}a e Innovaci\'{o}n, en  Ciencias de la Ingenier\'{i}a y Tecnolog\'{i}a. PSO con Nichos Interactivos y B\'{u}squedas locales con Quasi-Newton  (Morelia, México. September 2015 )
 \item Activities of X Anniversary of the Instituto Tecnológico Superior de Ciudad Hidalgo - ' Evolutionary computing applied to dynamical systems'.(Ciudad Hidalgo, México. October 2010).
\item Week of Research Projects FIE of the UMSNH - 'Gravitational Interactions Optimization ' ( Morelia, México. June 2010 ).
\item Week of Research Projects FIE of the UMSNH - 'Bifurcations Diagrams using Artificial Intelligence Tools'(Morelia, México. June 2009).
\end{innerlist}

\blankline

\textbf{Attended}
\begin{innerlist}
\item 5th HYCON2 Ph.D. School on Control of Networked and Large-Scale Systems and the EFFINET Ph.D. School on Control of Drinking Water Networks  (Lucca Italy, 1-5 of July 2013)
\item Java workshop in the 2nd Week of Computation and Systems. \textit{Morelia, Mexico (2006).}
\item Analysis and Object Oriented Design using UML  (Morelia Mexico,  8-12 of August 2011)

%\item Taller de Java en la 2da Semana de Sistemas y Computación (Morelia México, 2006)
%\item Week of Systems and computation in the Instituto Tecnológico de Morelia. \textit{Morelia, México (2006)}.
%\item 2da. Semana de Sistemas y Computaci\'on en el Instituto Tecnológico de Morelia (Morelia México, 2006).	
%\item Week of Systems and computation in the Instituto Tecnológico de Morelia. \textit{Morelia, México (2005).}
\end{innerlist}
%}}
\end{document}




%%%%%%%%%%%%%%%%%%%%%%%%%% End CV Document %%%%%%%%%%%%%%%%%%%%%%%%%%%%%

%%%%%%%%%%%%%%%%%%%%%%%%%% Git Instructions %%%%%%%%%%%%%%%%%%%%%%%%%%%%
%cd /Users/rodrigolopez/rdglpzCV/rdglpzEngCV
%git add rdglpzCVEng.tex
%git add rdglpzCVEng.pdf
%git commit -m "Changes"
%git push origin master
%%%%%%%%%%%%%%%%%%%%%%%%%%%%%%%%%%%%%%%%%%%%%%%%%%%%%%%%%%%%%%%%%%%%%%%%%


